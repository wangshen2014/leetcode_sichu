\subsubsection{内容简介}
Just For Testing

\begindot
\item 所有代码都是单一文件。这是因为一般OJ网站,提交代码的时候只有一个文本框,如果还是
按照标准做法,比如分为头文件.h和源代码.cpp,无法在网站上提交;

\item Shorter is better。能递归则一定不用栈;能用STL则一定不自己实现。

\item 不提倡防御式编程。不需要检查malloc()/new 返回的指针是否为nullptr;不需要检查内部函数入口参数的有效性。
\myenddot

本手册假定读者已经学过《数据结构》\footnote{《数据结构》,严蔚敏等著,清华大学出版社,
\myurl{http://book.douban.com/subject/2024655/}},
《算法》\footnote{《Algorithms》,Robert Sedgewick, Addison-Wesley Professional, \myurl{http://book.douban.com/subject/4854123/}}
这两门课,熟练掌握C++或Java。

\subsubsection{GitHub地址}
本书是开源的,GitHub地址:\myurl{https://github.com/xxxx}
